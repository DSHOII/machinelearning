\documentclass{article}
\input{Macros.tex}
\usepackage{amsmath}
\usepackage{geometry}
\usepackage{caption}
\usepackage{lipsum} 
\geometry{a4paper}
\usepackage[backend=biber,style=ieee]{biblatex}
\usepackage{comment}
\bibliography{ref}
\begin{document}
\title{Solution for ML F16 assignment 4}
\author{Jonathan Hansen fdg890}
\maketitle

\section{Finite Hypothesis Space}
To help us find the two hypothesis spaces, lets first look at the size of the
input space. With all the numbers in the range from \(0-100\) and two genders
the size of the input space is:

\[
\left\vert\chi\right\vert = \{0,\cdots,100\} \times \{male,female\} = 101 \cdot 2 = 202
\]
\
\
The output space is just the values \(-1\) and \(1\). Hence we can find the
number of all possible target functions and the size of the hypthesis set as

\[
\mathcal{H}_1 = 2^{101 \cdot 2} = 2^{202}
\]
\
\

To compute the size of the hypothesis space with the range approach we will
first find the number of possible ranges. Using the formula on positive
intervals from \cite{abu2012learning} this gives us

\[
m_{\mathcal{H}}(N)= \dfrac{1}{2}N^2 + \dfrac{1}{2}N + 1 = \dfrac{1}{2}\cdot 101^2
+ \dfrac{1}{2} \cdot 101 + 1 = 5152
\]
\
\

In this case any one of these ranges are given as the range where it is more
like than not, that male/females has minors. Since there is still two genders
this gives us an hypethesis set with size

\[
\mathcal{H}_2 = 5152^2
\]


\subsubsection*{Question 1.1}

\subsubsection*{Question 1.2}

\subsubsection*{Question 1.3}


\section{Occam's Razor}

\subsubsection*{Question 2.1}

\subsubsection*{Question 2.2}

\subsubsection*{Question 2.3}

\section{Logistic regression}


\subsubsection*{Question 3.1}


\subsubsection*{Question 3.2}


\subsubsection*{Question 3.3}


\printbibliography

\end{document}
