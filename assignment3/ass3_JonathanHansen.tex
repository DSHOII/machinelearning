\documentclass{article}
\input{Macros.tex}
\usepackage{amsmath}
\usepackage{geometry}
\usepackage{caption}
\usepackage{lipsum} 
\geometry{a4paper}
\usepackage[backend=biber,style=ieee]{biblatex}
\usepackage{comment}
\bibliography{ref}
\begin{document}
\title{Solution for ML F16 assignment 2}
\author{Jonathan Hansen fdg890}
\maketitle

\section{K-Nearest Neighbors}


\section{Markov’s inequality vs. Hoeffding’s inequality vs. binomial bound}
\subsubsection*{Question 2.1}
To bound the probability that \(\sum_{n=1}^{10} X_i\) we start by plugging
the given values into Markovs inequality:

\[
P\{S \ge 9\} \le \dfrac{\Em[S]}{9}
\]
\
Using the random variable we are given and linearity of expectation we get:

\[
P\{\displaystyle\sum_{n=1}^{10} X_i \ge 9\} \le \displaystyle\sum_{n=1}^{10}
\dfrac{\Em[X_i]}{9}
\]
\
\
Since we know that \(X_1\),...,\(X_{10}\) are i.i.d Bernoulli random variables
with bias \(\frac{1}{2}\) the expected value sums to \(\frac{5}{9}\).

\subsubsection*{Question 2.2}
Focusing at first at the probability side of Hoeffding's inequeality we can use
the information given, and the result of the sum of expected valus from above,
to determine \(\epsilon\):

\[
\mathbb{P}\left\{\displaystyle\sum_{n=1}^{10} X_i -
\Em\bigg[\displaystyle\sum_{n=1}^{10} X_i \bigg] \ge \epsilon \right\} \Leftrightarrow
\mathbb{P}\left\{\displaystyle\sum_{n=1}^{10} X_i \ge \epsilon + 5 \right\} \Rightarrow
\epsilon = 4
\]
\
\
With this result we can find the bound we are asked for, since we know that
\(a_i=0\) and \(b_i=1\):

\[
e^{-2(4)^2/\sum_{n=1}^{10}(1-0)^2} = e^{-\frac{16}{5}}
\]

\subsubsection*{Question 2.3}
We can use the normal binomial


\subsubsection*{Question 2.4}


\section{Probability Theory: Sample Space}
\section{Probability Theory: Properties of Expectation}
\section{Probability Theory: Complements of Events}

\printbibliography

\end{document}
