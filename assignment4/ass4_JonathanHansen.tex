\documentclass{article}
\usepackage{graphicx,amsmath,latexsym}
\usepackage{color,psfrag,boxedminipage,amssymb}
%\usepackage[latin1]{inputenc}
%\usepackage[swedish]{babel}
%\usepackage{draftcopy}
\usepackage{tikz}
\usepackage{pgf}
\usetikzlibrary{positioning,arrows}

\newcommand{\st}{\ensuremath{\boldsymbol{s}_k(\boldsymbol{\theta})}}
\newcommand{\St}{{\mathbf S}\ensuremath{(\boldsymbol{\theta})}}
\newcommand{\phibf}{\ensuremath{\boldsymbol{\phi}}}
\newcommand{\varphibf}{\ensuremath{\boldsymbol{\varphi}}}
\def \phibfh{\ensuremath{\hat{\boldsymbol{\phi}}}}
\renewcommand{\sp}{\ensuremath{\boldsymbol{s}(\boldsymbol{\phi})}}
\newcommand{\hp}{\ensuremath{\boldsymbol{h}(\boldsymbol{\phi})}}
\newcommand{\thk}{\ensuremath{\boldsymbol{\theta}_k}}
\renewcommand{\th}{\ensuremath{\boldsymbol{\theta}}}
\newcommand{\alp}{\ensuremath{\boldsymbol{\alpha}}}
\def \thh{\ensuremath{\hat{\boldsymbol{\theta}}}}
\def \epsbf{\ensuremath{\boldsymbol{\epsilon}}}
\def \tht{\tilde{\th}}
\newcommand{\mubf}{\ensuremath{\boldsymbol{\mu}}}

\newcommand{\ctft}{\buildrel{{\cal F}}\over{\longleftrightarrow}}
\newcommand{\defin}{\buildrel{\triangle}\over{=}}

\def \Em{{\mathbb{E}}}



\def \abf{{\mathbf a}}
\def \Abf{{\mathbf A}}
\def \bbf{{\mathbf b}}
\def \Bbf{{\mathbf B}}
\def \cbf{{\mathbf C}}
\def \Cbf{{\mathbf C}}
\def \dbf{{\mathbf d}}
\def \Dbf{{\mathbf D}}
\def \ebf{{\mathbf e}}
\def \Ebf{{\mathbf E}}
\def \fbf{{\mathbf f}}
\def \Fbf{{\mathbf F}}
\def \gbf{{\mathbf g}}
\def \Gbf{{\mathbf G}}
\def \hbf{{\mathbf h}}
\def \Hbf{{\mathbf H}}
\def \ibf{{\mathbf i}}
\def \Ibf{{\mathbf I}}
\def \jbf{{\mathbf j}}
\def \Jbf{{\mathbf J}}
\def \kbf{{\mathbf k}}
\def \Kbf{{\mathbf K}}
\def \lbf{{\mathbf l}}
\def \Lbf{{\mathbf L}}
\def \mbf{{\mathbf m}}
\def \Mbf{{\mathbf M}}
\def \nbf{{\mathbf n}}
\def \Nbf{{\mathbf N}}
\def \obf{{\mathbf o}}
\def \Obf{{\mathbf O}}
\def \pbf{{\mathbf p}}
\def \Pbf{{\mathbf P}}
\def \qbf{{\mathbf q}}
\def \Qbf{{\mathbf Q}}
\def \rbf{{\mathbf r}}
\def \Rbf{{\mathbf R}}
\def \sbf{{\mathbf s}}
\def \Sbf{{\mathbf S}}
\def \tbf{{\mathbf t}}
\def \Tbf{{\mathbf T}}
\def \ubf{{\mathbf u}}
\def \Ubf{{\mathbf U}}
\def \vbf{{\mathbf v}}
\def \Vbf{{\mathbf V}}
\def \wbf{{\mathbf w}}
\def \Wbf{{\mathbf W}}
\def \xbf{{\mathbf x}}
\def \Xbf{{\mathbf X}}
\def \ybf{{\mathbf y}}
\def \Ybf{{\mathbf Y}}
\def \zbf{{\mathbf z}}
\def \Zbf{{\mathbf Z}}
\def \0bf{{\mathbf 0}}

\def \Emean{\mathbb{E}}

\def \Sibf{{\mathbf \Sigma}}
\def \xbbf{\mathbf{\bar{x}}}
\def \etr{\mbox{etr}}
\def \tr{\mbox{tr}}
\def \Tr{\mbox{Tr}}
\def \Cov{\mbox{Cov}}
\def \cost{\mbox{cost}}
\def \diag{\mbox{diag}}
\def \Lambf{{\mathbf{\Lambda}}}
\def \Gambf{{\mathbf{\Gamma}}}
\def \Sigbf{{\mathbf \Sigma}}
\newcommand{\rhobf}{\ensuremath{\boldsymbol{\rho}}}
\newcommand{\lambf}{\ensuremath{\boldsymbol{\lambda}}}
\newcommand{\nubf}{\ensuremath{\boldsymbol{\nu}}}

\newcounter{examplenr}[section]
\renewcommand{\theexamplenr}{\arabic{examplenr}}%{\thesection.\arabic{examplenr}}
\newenvironment{example}[1]{\vskip \baselineskip
\refstepcounter{examplenr}\noindent{{\bf
Example~\theexamplenr}\hskip .5em #1\\} }{\hrulefill $\Box$
 \vskip\baselineskip}



\newcounter{theoremnr}[section]
%\renewcommand{\thetheoremnr}{\thesection.\arabic{theoremnr}}
\renewcommand{\thetheoremnr}{\arabic{theoremnr}}
\newtheorem{A}[theoremnr]{Theorem}
\newcounter{theoremProofnr}[section]
%\renewcommand{\thetheoremProofnr}{\thesection.\arabic{theoremProofnr}}
\renewcommand{\thetheoremProofnr}{\arabic{theoremProofnr}}
\newtheorem{B}[theoremProofnr]{Proof of Theorem}
\newcounter{corollarynr}[section]
\renewcommand{\thecorollarynr}{\thesection.\arabic{corollarynr}}
\newtheorem{C}[corollarynr]{Corollary}
\newcounter{Lemmanr}[section]
\renewcommand{\theLemmanr}{\thesection.\arabic{Lemmanr}}
\newtheorem{D}[Lemmanr]{Lemma}
\newcounter{LemmaProofnr}[section]
\renewcommand{\theLemmaProofnr}{\thesection.\arabic{LemmaProofnr}}
\newtheorem{E}[LemmaProofnr]{Proof of Lemma}

\newcounter{propositionnr}[section]
\renewcommand{\thepropositionnr}{\arabic{propositionnr}}
\newtheorem{F}[propositionnr]{Proposition}
\newcounter{propositionProofnr}[section]
\renewcommand{\thepropositionProofnr}{\arabic{propositionProofnr}}
\newtheorem{G}[propositionProofnr]{Proof of proposition}

\newenvironment{working}{\color{blue}\sffamily\em}{}
\newenvironment{forslag}{\color{red}\sffamily\em}{}

\newcounter{definnr}[section]
\renewcommand{\thedefinnr}{\arabic{definnr}}
\newtheorem{J}[definnr]{Definition}



\usepackage{amsmath}
\usepackage{geometry}
\usepackage{caption}
\usepackage{lipsum} 
\geometry{a4paper}
\usepackage[backend=biber,style=ieee]{biblatex}
\usepackage{comment}
\bibliography{ref}
\begin{document}
\title{Solution for ML F16 assignment 4}
\author{Jonathan Hansen fdg890}
\maketitle

\section{Finite Hypothesis Space}
\subsubsection*{Question 1.1}

To help us find the two hypothesis spaces, lets first look at the size of the
input space. With all the numbers in the range from \(0-100\) and two genders
the size of the input space is:

\[
\left\vert\chi\right\vert = \{0,\cdots,100\} \times \{male,female\} = 101 \cdot 2 = 202
\]
\
\
The output space is just the values \(-1\) and \(1\). Hence we can find the
number of all possible target functions and the size of the hypthesis set as

\[
\mathcal{H}_1 = 2^{101 \cdot 2} = 2^{202}
\]
\
\

To compute the size of the hypothesis space with the range approach we will
first find the number of possible ranges. Using the formula on positive
intervals from \cite{abu2012learning} this gives us

\[
m_{\mathcal{H}}(N)= \dfrac{1}{2}N^2 + \dfrac{1}{2}N + 1 = \dfrac{1}{2}\cdot 101^2
+ \dfrac{1}{2} \cdot 101 + 1 = 5152
\]
\
\

In this case any one of these ranges are given as the range where it is more
like than not, that male/females has minors. Since there is still two genders
this gives us an hypethesis set with size

\[
\mathcal{H}_2 = 5152^2
\]


\section{Logistic regression}
All references for equastions in this section points to the course textbook
\cite{abu2012learning}.

\subsubsection*{Question 3.1}
From \((3.7)\) we know the formula for conditional properties for
\(y\) given \(x\), when \(y \in \{\pm 1\}\):

\begin{equation}
  P(y \vert X) =
  \begin{cases}
    h(X) & \text{for}\ y=+1 \\
    1-h(X) & \text{for}\ y=-1
  \end{cases}
\end{equation}
\
\
On page 91 we are given a quantity to minimize that is equivalent to the method
of maximum likelihood:

\begin{equation}
\dfrac{1}{N}\displaystyle\sum_{n=1}^{N} ln\bigg(\frac{1}{P(y_n \vert X_n} \bigg)
\end{equation}
\
\

If we split the quantity in \((2)\) into two cases based on whether \(y=+1\) or
\(y=-1\) and substitute the formulas from \((1)\) into this expression, we
almost have what we want to show. The final thing to do is simply to realize
that scaling with the constant \(N\) does not change the minimum or maximum
value why we can ommit this.
\vspace{1cm}


\subsubsection*{Question 3.3}
I have implemented the logistic regression classifier in \texttt{Python3}. I
have made substantial use of the libraries \texttt{Pandas} and
\texttt{Numpy}. All the code is gathered in a single script called
\texttt{logreg.py}. Running it using "python3 logreg.py" from the terminal
reports the required values.

I have implemented the logistic regression using the "steepest descent"
approach. Building the affine linear model gives the paramters shown in the
table below. Using these parameters to perform a classication yields the 0-1
losses that are also shown in the table.

\begin{figure}[h]
\centering
\begin{tabular}{ |p{5cm}||p{5cm}|  }
 \hline
 \multicolumn{2}{|c|}{Parameters of the affine linear model and loss}\\
 \hline
 weights   & [0.68113276 \ \ \ -2.44553796] \\
 b  & -2.94487779088 \\
 Training data loss & 9.6774 \% \\
 Test data loss & 7.6923 \% \\
 \hline
\end{tabular}
\end{figure}


\printbibliography

\end{document}
